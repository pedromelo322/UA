\documentclass[a4paper,12pt]{article}

% --- Pacotes Principais ---
\usepackage[utf8]{inputenc}
\usepackage[T1]{fontenc}
\usepackage[portuguese]{babel} % Configuração para Português de Portugal
\usepackage{graphicx}
\usepackage{geometry}
\usepackage{hyperref}
\usepackage{fancyhdr}
\usepackage{listings}
\usepackage{xcolor}
\usepackage{float}
\usepackage{booktabs}
\usepackage{caption}

% --- Configuração de Margens ---
\geometry{top=2.5cm, bottom=2.5cm, left=2.5cm, right=2.5cm}

% --- Configuração de Links ---
\hypersetup{
    colorlinks=true,
    linkcolor=black,
    filecolor=magenta,      
    urlcolor=blue,
    citecolor=black,
}

% --- Configuração de Code Snippets ---
\definecolor{codegray}{rgb}{0.95,0.95,0.95}
\lstset{
    backgroundcolor=\color{codegray},
    basicstyle=\ttfamily\small,
    breakatwhitespace=false,         
    breaklines=true,                 
    captionpos=b,                    
    keepspaces=true,                 
    numbers=left,                    
    numbersep=5pt,                  
    showspaces=false,                
    showstringspaces=false,
    showtabs=false,                  
    tabsize=2
}

% --- Informações do Título ---
\title{
    \vspace{-2cm}
    \includegraphics[width=0.4\textwidth]{example-image-a} \\ % Substituir pelo logo da UA se tiveres
    \vspace{1cm}
    \textbf{Arquiteturas de Comunicação}\\
    \large Projeto: Rede Datacenter DC4ALL com Suporte a Multi-Clientes
}
\author{
    \textbf{Autores:} \\
    Nome do Aluno 1 (Mec. XXXXX) \\
    Nome do Aluno 2 (Mec. XXXXX) \\
    \\
    \textit{Professores:} \\
    Rui Aguiar \\
    Paulo Salvador
}
\date{Janeiro de 2025}

% --- Início do Documento ---
\begin{document}

\maketitle
\thispagestyle{empty}
\newpage

% --- Resumo ---
\begin{abstract}
Este relatório descreve o projeto, configuração e validação de uma infraestrutura de rede para a operadora de Datacenters DC4ALL LLC. O objetivo principal foi interligar dois datacenters geograficamente distribuídos (Porto e Lisboa) através de uma rede proprietária baseada em MPLS e DiffServ. Foram implementadas redes privadas Ethernet (EVPN) para dois grandes clientes (L1 e L2) com requisitos distintos. Para o Cliente L1, foi assegurada uma largura de banda garantida e alta resiliência a falhas. Para o Cliente L2, foi implementada diferenciação de tráfego Layer 2 com políticas de \textit{Assured Forwarding}. A solução utiliza routers Cisco C7200 no core e contentores Linux com FRRouting nos datacenters, validando a interoperabilidade e robustez da arquitetura proposta.
\end{abstract}
\newpage

% --- Índice ---
\tableofcontents
\newpage

% --- Conteúdo ---

\section{Introdução}
A empresa DC4ALL LLC, operadora de datacenters e Sistema Autónomo (AS 22900), requer a implementação de uma infraestrutura de comunicação robusta para interligar os seus centros de dados em Lisboa e no Porto.

O âmbito deste projeto consiste no desenho técnico e configuração de uma rede que suporte a interligação transparente de servidores virtuais e \textit{bare-metal} para múltiplos clientes empresariais. A infraestrutura baseia-se no bloco de endereçamento IPv4 10.0.0.0/22 e deve satisfazer requisitos estritos de isolamento de tráfego e qualidade de serviço (QoS)[cite: 7, 9, 15].

\section{Arquitetura da Rede e Planeamento}

\subsection{Topologia}
A topologia da rede é composta por dois sites principais (Datacenters) interligados por um núcleo de rede (Core).
\begin{itemize}
    \item \textbf{Core Network:} Composta pelos routers \textit{Core 1} e \textit{Core 2} (Cisco C7200), que fornecem redundância e conetividade entre as cidades.
    \item \textbf{Edge Routers:} Routers de fronteira localizados em Lisboa e Porto (Cisco C7200).
    \item \textbf{Datacenter Fabric:} Switches multi-camada ($S^*$, $L^*$) implementados com contentores Linux correndo FRRouting[cite: 27].
\end{itemize}

\subsection{Endereçamento e Clientes}
Foram definidos os seguintes parâmetros para os clientes[cite: 16, 17, 18]:

\begin{table}[H]
\centering
\begin{tabular}{|c|c|c|c|}
\hline
\textbf{Cliente} & \textbf{VLAN / Rede} & \textbf{Sub-rede} & \textbf{Requisitos Específicos} \\
\hline
\multirow{3}{*}{L1} & VLAN 10 & 10.10.0.0/22 & \multirow{3}{*}{\shortstack{EVPN Privada, 10Mbps garantidos, \\ Alta Resiliência}} \\
 & VLAN 20 & 10.20.0.0/22 & \\
 & VLAN 30 & 10.30.0.0/22 & \\
\hline
L2 & N/A (LAN) & 10.40.0.0/22 & \shortstack{EVPN Privada, DiffServ \\ (Assured Forwarding), máx 10Mbps} \\
\hline
\end{tabular}
\caption{Requisitos de Endereçamento e Serviço dos Clientes}
\end{table}

\section{Implementação Técnica}

\subsection{Conectividade Base e IGP (Underlay)}
A conectividade base (Underlay) foi estabelecida utilizando o protocolo OSPF (Open Shortest Path First). Todos os routers Cisco e switches FRR anunciam as suas interfaces de \textit{loopback} e ligações ponto-a-ponto.

\textit{[Sugestão: Inserir aqui um excerto da configuração OSPF de um router Cisco e de um FRR, ou um output de 'show ip ospf neighbor']}

\subsection{Infraestrutura MPLS e BGP (Overlay)}
Sobre a rede IP base, foi configurado MPLS (Multiprotocol Label Switching) no core para permitir engenharia de tráfego e suporte a VPNs. O protocolo BGP (Border Gateway Protocol) foi utilizado para a troca de rotas de EVPN entre os datacenters.
\begin{itemize}
    \item \textbf{LDP:} Utilizado para distribuição de etiquetas no core.
    \item \textbf{MP-BGP EVPN:} Configurado entre os dispositivos de fronteira para transportar a informação de reachability MAC/IP dos clientes L1 e L2[cite: 34].
\end{itemize}

\subsection{Cliente L1: Resiliência e Largura de Banda}
Para o Cliente L1, que exige uma largura de banda garantida de 10Mbps e alta resiliência[cite: 21, 22], foi implementada uma solução baseada em engenharia de tráfego.

Dado que os routers Cisco C7200 têm limitações no encaminhamento baseado em VNIs VXLAN, a diferenciação de tráfego foi realizada com base nos endereços IP de origem/destino ou portos UDP encapsulados.
\begin{enumerate}
    \item \textbf{Reserva de Banda:} Utilização de RSVP-TE ou \textit{Policy Based Routing} (PBR) para direcionar o tráfego do L1 para túneis específicos com reserva de recursos.
    \item \textbf{Resiliência:} Configuração de túneis de backup (FRR - Fast Reroute) para garantir a recuperação rápida em caso de falha de um link no core.
\end{enumerate}

\subsection{Cliente L2: Diferenciação de Tráfego (DiffServ)}
O Cliente L2 requereu uma política de \textit{Assured Forwarding}[cite: 25]. A implementação seguiu a arquitetura DiffServ:
\begin{itemize}
    \item \textbf{Classificação e Marcação:} No ingresso da rede (switches FRR ou Routers de Borda), o tráfego proveniente da rede 10.40.0.0/22 foi marcado com valores DSCP apropriados (ex: AF31 ou AF21).
    \item \textbf{Policiamento e Shaping:} Nos routers do core, foram aplicadas \textit{Service Policies} para garantir que este tráfego tem prioridade, mas está limitado a 10 Mbps ("guaranteed up to").
\end{itemize}

\section{Testes e Validação}

\subsection{Verificação de Conectividade (Ping e Traceroute)}
Foram realizados testes de conectividade entre as diferentes racks e datacenters para ambos os clientes.
\begin{itemize}
    \item \textbf{Cliente L1:} Sucesso na comunicação inter-VLAN e intra-VLAN entre Porto e Lisboa.
    \item \textbf{Cliente L2:} Sucesso na comunicação da LAN estendida.
\end{itemize}

\textit{[Sugestão: Inserir screenshot de um ping entre hosts do Cliente L1 localizados em sites diferentes]}

\subsection{Validação de Largura de Banda (Iperf)}
Para validar os requisitos de 10Mbps:
\begin{figure}[H]
    \centering
    % \includegraphics[width=0.8\textwidth]{iperf_result.png} 
    \caption{Exemplo de teste Iperf demonstrando o limite de largura de banda.}
    \label{fig:iperf}
\end{figure}
Os testes demonstraram que o tráfego do Cliente L1 mantém a estabilidade nos 10Mbps mesmo sob carga, enquanto o Cliente L2 respeita o teto máximo definido pela política de QoS.

\subsection{Teste de Resiliência}
Ao desativar a ligação principal entre o Router Lisboa e o Core 1, o tráfego do Cliente L1 convergiu automaticamente para o caminho alternativo via Core 2, sem perda significativa de pacotes, validando a alta resiliência exigida.

\section{Conclusão}
O projeto permitiu desenhar e validar com sucesso uma arquitetura de rede de duplo datacenter complexa. A utilização combinada de routers Cisco e contentores FRR demonstrou a viabilidade de soluções híbridas.
Todos os objetivos foram cumpridos: as EVPNs garantiram o isolamento dos clientes L1 e L2; as políticas de MPLS-TE asseguraram os 10Mbps vitais para o L1; e o modelo DiffServ geriu corretamente a diferenciação de tráfego do L2. A limitação dos routers Cisco relativa aos VNIs foi contornada através do mapeamento de tráfego baseado em portos/IPs, cumprindo as restrições do enunciado.

% --- Bibliografia (Opcional ou baseada nos slides da cadeira) ---
% \begin{thebibliography}{9}
% \bibitem{rfc7432} BGP MPLS-Based Ethernet VPN. RFC 7432.
% \end{thebibliography}

\end{document}
