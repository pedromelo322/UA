\documentclass{report}
\usepackage[T1]{fontenc} % Fontes T1
\usepackage[utf8]{inputenc} % Input UTF8
\usepackage[backend=biber, style=ieee]{biblatex} % para usar bibliografia
\usepackage{csquotes}
\usepackage[portuguese]{babel} %Usar língua portuguesa
\usepackage{blindtext} % Gerar texto automaticamente
\usepackage[printonlyused]{acronym}
\usepackage{hyperref} % para autoref
\usepackage{graphicx}
\usepackage{indentfirst}

\bibliography{bibliografia}

\usepackage{graphicx}
\usepackage{adjustbox}

\begin{document}
%%
% Definições
%
\def\titulo{Projeto Final de Laboratórios de Informática}
\def\data{DATA}
\def\autores{Henrique Ferreira, Martina Duque, Pedro Melo, Sofia Marrafa}
\def\autorescontactos{(113600)ferreira.manuel.henrique04@ua.pt, (113261) martina.duque18@ua.pt\\(114208) pedro.m.melo@ua.pt, (114591) sofiamarrafa@ua.pt}
\def\versao{VERSAO}
\def\departamento{Dept. de Eletrónica, Telecomunicações e Informática}
\def\empresa{Universidade de Aveiro}
\def\logotipo{ua.pdf}
%
%%%%%% CAPA %%%%%%
%
\begin{titlepage}

\begin{center}
%
\vspace*{50mm}
%
{\Huge \titulo}\\ 
%
\vspace{10mm}
%
{\Large \empresa}\\
%
\vspace{10mm}
%
{\LARGE \autores}\\ 
%
\vspace{30mm}
%
\begin{figure}[h]
\center
\includegraphics{\logotipo}
\end{figure}
%
\vspace{30mm}
\end{center}
%


\end{titlepage}

%%  Página de Título %%
\title{%
{\Huge\textbf{\titulo}}\\
{\Large \departamento\\ \empresa}
}
%
\author{%
    \autores \\
    \autorescontactos
}
%
\date{\today}
%
\maketitle

\pagenumbering{roman}

%%%%%% RESUMO %%%%%%
\begin{abstract}

Neste projeto, no âmbito da disciplina de Laboratórios de Informática, foi proposto o planeamento, preparação e implementação de uma página web para gerir imagens num perfil próprio. Os utilizadores terão a capacidade de criar uma conta pessoal, fazer upload das suas próprias imagens, atribuir-lhes nome e data, e até mesmo editá-las. Além disso, poderão visualizar as imagens individualmente, comentar e votar nelas.

O projeto engloba diversas tecnologias e componentes, tais como uma interface web desenvolvida com HTML, CSS e JavaScript. Além disso, utilizaremos uma aplicação web desenvolvida em Python para gerir a base de dados e o processamento de imagem e JavaScript Object Notation (JSON) gerenciando assim a comunicação entre dados. A persistência dos dados será realizada através de um banco de dado SQLite3, enquanto os ficheiros de imagem serão armazenados no sistema de ficheiros.


É através da interface web, que os utilizadores terão a possibilidade de interagir com a aplicação e será a aplicação web em Python, a responsável por fornecer os recursos necessários para a criação e gestão das contas de utilizadores, assim como para a realização das operações de upload e visualização de fotos existentes na galeria. Para além do mais, esta irá ainda fornecer funcionalidades para o processamento das mesmas, permitindo aos utilizadores a edição das suas imagens de acordo com as suas preferências.


Esta aplicação web foi desenvolvida com um sistema de registro e login de modo a que, de uma forma segura, os utilizadores possam carregar imagens com informações como autor, nome e data de carregamento. Além disso, estes podem visualizar as imagens existentes, filtrá-las por autor, ver detalhes de cada imagem, adicionar comentários e votar nas imagens.
A escolha do banco de dados SQLite3 como meio de persistência de dados garante um armazenamento seguro e eficiente das informações relacionadas às contas dos utilizadores, imagens, comentários e das interações.



\end{abstract}

%%%%%% Agradecimentos %%%%%%
% Segundo glisc deveria aparecer após conclusão...
\tableofcontents
% \listoftables     % descomentar se necessário
% \listoffigures    % descomentar se necessário


%%%%%%%%%%%%%%%%%%%%%%%%%%%%%%%
\clearpage
\pagenumbering{arabic}

%%%%%%%%%%%%%%%%%%%%%%%%%%%%%%%%
\chapter{Introdução}
\label{chap.introducao}

Neste projeto, será apresentado o planeamento, preparação e implementação de uma página web para gerir imagens num perfil próprio na qual os utilizadores terão a capacidade de criar uma conta pessoal, fazer upload das suas próprias imagens, atribuir-lhes nome e data, e até mesmo editá-las. 

Este documento encontra-se dividido em 6 capítulos. Depois desta introdução,
no \autoref{chap.interface} é apresentada a interface web seguida,
no \autoref{chap.aplicaçao} da página Web. Para além destes, têm se ainda os capítulos \autoref{chap.persistencia} e \autoref{chap.process} onde são apresentados os capítulos da persistência e do processamento de imagem.
Finalmente, no \autoref{chap.conclusao} são apresentadas
as conclusões do trabalho.


\chapter{Interface Web}
\label{chap.interface}

Esta componente é a grande responsável pelas várias funcionalidades do website, sendo composta por diversos ficheiros HTML, CSS, JavaScript dando assim origem a uma única interface para interação com o sistema permitindo assim o registo de novos utilizadores e login dos mesmos.

Após o registo de cada utilizador, estes poderão fazer login na sua própria conta, sendo estes reencaminhados para a página principal, onde será possível visualizar a sua própria galeria (bem como a de outros utilizadores) e dar upload das suas próprias imagens.

\begin{figure}[h]
  \centering
  \begin{adjustbox}{center}
    \includegraphics[width=150mm]{singin.png}
  \end{adjustbox}
  \caption{Registar conta}
  \label{fig:singin}
\end{figure}

\begin{figure}[h]
  \centering
  \begin{adjustbox}{center}
    \includegraphics[width=150mm]{login.png}
  \end{adjustbox}
  \caption{Login}
  \label{fig:singin}
\end{figure}

Acedendo à opção de upload, é possível carregar qualquer foto que utilizador pretenda guardar, sendo ainda permitido atribuir nome e juntamente serão carregadas as informações de data de carregamento e autor, que corresponderá à pessoa que irá dar upload da imagem.

Após isto, ao abrir a aba de Gallery, o utilizador poderá encontrar a foto anteriormente carregada pelo mesmo, bem como a de outros utilizadores, podendo assim, ao selecionar qualquer foto, comentar na mesma e interagir de forma positiva ou negativa.

Caso o utilizador deseje ainda alterar uma foto da sua autoria, este pode alterar algumas das propriedades das imagens tais como luminusidade, contraste e cores, sendo ainda possível alterar a imagem original para a imagem modificada.

Para além destas páginas, existe ainda uma página About, na qual cada utilizador poderá visualizar os autores da aplicação.


\begin{figure}[h]
  \centering
  \begin{adjustbox}{center}
    \includegraphics[width=120mm]{Upload.png}
  \end{adjustbox}
  \caption{Upload}
  \label{fig:singin}
\end{figure}


\begin{figure}[h]
  \centering
  \begin{adjustbox}{center}
    \includegraphics[width=120mm]{Gallery.png}
  \end{adjustbox}
  \caption{Gallery}
  \label{fig:singin}
\end{figure}


\begin{figure}[h]
  \centering
  \begin{adjustbox}{center}
    \includegraphics[width=120mm]{process.png}
  \end{adjustbox}
  \caption{Process}
  \label{fig:singin}
\end{figure}




\begin{figure}[h]
  \centering
  \begin{adjustbox}{center}
    \includegraphics[width=120mm]{About.png}
  \end{adjustbox}
  \caption{About}
  \label{fig:singin}
\end{figure}



\chapter{Aplicação Web}
\label{chap.aplicaçao}
A aplicação web utilizada para o bom funcionamento deste site foi desenvolvido em Python e serviu como camada intermediária entre a interface web e o banco de dados. Assim, será  este que nos irá servir os conteúdos estáticos já referidos anteriormente (HTML, CSS e JavaScript) permitindo a navegação pelos diversos componentes da interface. Este fornece métodos para a navegação entre os diferentes componentes e também uma interface programática para obtenção e inserção de informações relacionadas à imagens, autores, comentários e votos. Os métodos expostos pela aplicação web retornam objetos JSON, utilizados na junção de dados que permite uma comunicação entre dados estruturados no servidor de CherryPy e os clientes HTTP.

\chapter{Persistência}
\label{chap.persistencia}

Este componente é composto por métodos que permitem o registo de informação numa base de dados relacional e a obtenção de informações da mesma. Os métodos serão utilizados pela Aplicação Web para registar a informação relativa às imagens existentes no sistema, bem como os comentários e os votos.


A base de dados utilizada é a SQLite3, localizada no mesmo diretório da aplicação. Os ficheiros contendo as imagens são armazenados no sistema de ficheiros (no diretório uploads), sendo que a base de dados relacional terá informação sobre as imagens. Isto é, a base de dados tem informação sobre a data de carregamento da imagem e respetiva informação (nome, autor, comentários e votos).



\chapter{Processador de Imagens}
\label{chap.process}

O componente de processamento de imagens permite simular algoritmos de manipulação de imagens. Os algoritmos foram implementados em um arquivo Python separado e são importados na aplicação principal. Os resultados do processamento são armazenados em um diretório temporário e os arquivos são removidos quando não são mais necessários.



\chapter{Conclusões}
\label{chap.conclusao}
Em suma, este projeto requer a interação entre várias tecnologias e componentes, incluindo uma interface web desenvolvida com HTML, CSS e JavaScript, e uma aplicação web em Python para o relacionamento das funcionalidades principais, como manipulação de imagens, gestão de contas e persistência de dados. O uso do banco de dados SQLite3 assegura ainda a consistência e confiabilidade das informações armazenadas.

\chapter*{Contribuições dos autores}

A realização deste trabalho culminou na contribuição de todos os elementos. Todos os integrantes do grupo esforçaram-se e dedicaram várias horas para o melhor funcionamento possível do web site.

Assim foi atribuída a mesma percentagem para cada elemento.
\vspace{10pt}

\autores : 25\%, 25\%, 25\%, 25\%\\

\textbf{Repositório GitHub:} labi2023g22

%%%%%%%%%%%%%%%%%%%%%%%%%%%%%%%%%

%%%%%%%%%%%%%%%%%%%%%%%%%%%%%%%%%
\printbibliography

\end{document} 